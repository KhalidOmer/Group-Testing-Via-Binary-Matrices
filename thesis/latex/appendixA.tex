% Note LaTeX Produce an error when \\ is used in normal text 
\chapter{}
\section{Norm}

A norm of a vector space is a mapping that associates with each vector $ v $ a real number $||v||$ such that the following properties are satisfied for all vectors $ u, v $ and all scalars $ c \in \mathbb{R} $; 
\begin{enumerate}
	\item $ ||v|| \geq 0 $ and $ ||v|| = 0 $ if and only if $ v = 0 $.
	\item $||cv|| = |c| ||v||$.
	\item $||u+v|| \leq ||u|| + ||v||$
\end{enumerate}
The general norm in $ \mathbb{R} $ is defined as follows: 
\begin{equation}\label{key}
||v||_p = (|v_1|^p + ... + |v_n|^p)^{\frac{1}{p}}
\end{equation}
The first norm $ ||v||_1 $ also known as the taxicab is then obtained by setting $ p = 1 $ \begin{equation}\label{key}
||v||_1 = |v_1| + |v_2| + ... + |v_n|
\end{equation}
The second norm of the usual euclidean norm is obtained by setting $ p = 2 $ \begin{equation}\label{key}
||v||_2 = (|v_1|^2+...+|v_n|^2)^{\frac{1}{2}}
\end{equation} 

While as for the $ ||v||_0 $ or the pseudo norm is actually a comparison tool rather being the literal norm defined above, the $ ||v||_0 $ or $ l_0 $ as refereed to in the text  between counts the number of nonzero components in a vector. 
\subsection{Matrix norms induced by  p-norms}
A matrix norm on $ M_{nn} : V^{m\times n} \longrightarrow \mathbf{R}$ is a mapping that associates with $ m\times n $ matrix $ A $ a real number $ ||A|| $ called the norm of $ A $, such that the following properties are satisfied for all $ m \times n $ matrices and scalar $ \alpha \in V$ \begin{enumerate}
	\item $ ||A|| \geq 0 $.
	\item $ ||A|| = 0 \Longleftrightarrow A = 0_{m\times n} $. 
	\item $ ||\alpha A|| = |\alpha| ||A||$.
	\item $ ||A+B|| \leq ||A|| + ||B|| $.
\end{enumerate}

For $ 1 \geq p \leq \infty$ the corresponding operator norm is \begin{equation} ||A||_p = sup_{x \neq 0} \frac{||Ax||_p}{||x||_p} \end{equation} 
For the special cases $ p = 1,2$ the matrix norm is defined as follows: \begin{equation}
||A||_1 = max_{1\leq j \leq n} \sum_{i =1}^{m} |a_{ij} |
\end{equation}
\begin{equation}
||A||_2 = \sigma_{max} (A) 
\end{equation} where $\sigma_{max} (A)$ is the square root of the largest eigen value of $ AA^{T} $.

\section{Python code}

The codes that are used in this thesis are available at: 


The file README.md contains a brief description of the codes and what they do as well as how to
run them.



