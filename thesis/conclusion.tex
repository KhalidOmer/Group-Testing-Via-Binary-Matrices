% Note LaTeX Produce an error when \\ is used in normal text 
\chapter*{Conclusion}
\addcontentsline{toc}{chapter}{Conclusion}

The idea of group testing is that we divide the population into groups. For each group then mix the samples drawn from each individual in the group and test the mixture. If the result of the test comes out negative then the group is free of the virus and if the result comes out positive then at least one individual of the group is infected. 


The goal of group testing research is to minimize the number of tests required to identify infected individuals in a population.  The problem is equivalent to solve a system of linear equations, with the number of equations being less the number of the unknowns.  Generally, there are two types of group testings, adaptive tests and non-adaptive test. In this work we consider the non-adaptive tests, where the tests are fixed in advance and conducted in parallel.


We present two topics here, first, the construction of efficient test designs. With the perfect design we can reduce the cost of detection, time and storage wise.  The second we review an algorithm that works on retrieving the missing information.  Here we verify that The Hard thresholding algorithms are robust and accurate in the decoding process.  